\documentclass[]{book}

%These tell TeX which packages to use.
\usepackage{array,epsfig}
\usepackage{amsmath}
\usepackage{amsfonts}
\usepackage{amssymb}
\usepackage{amsxtra}
\usepackage{amsthm}
\usepackage{mathrsfs}
\usepackage{color}
\usepackage{enumitem}
%Here I define some theorem styles and shortcut commands for symbols I use often
\theoremstyle{definition}
\newtheorem{defn}{Definition}
\newtheorem{thm}{Theorem}
\newtheorem{cor}{Corollary}
\newtheorem*{rmk}{Remark}
\newtheorem{lem}{Lemma}
\newtheorem*{joke}{Joke}
\newtheorem{ex}{Example}
\newtheorem*{soln}{Solution}
\newtheorem{prop}{Proposition}

\newcommand{\lra}{\longrightarrow}
\newcommand{\ra}{\rightarrow}
\newcommand{\surj}{\twoheadrightarrow}
\newcommand{\graph}{\mathrm{graph}}
\newcommand{\bb}[1]{\mathbb{#1}}
\newcommand{\Z}{\bb{Z}}
\newcommand{\Q}{\bb{Q}}
\newcommand{\R}{\bb{R}}
\newcommand{\C}{\bb{C}}
\newcommand{\N}{\bb{N}}
\newcommand{\M}{\mathbf{M}}
\newcommand{\m}{\mathbf{m}}
\newcommand{\MM}{\mathscr{M}}
\newcommand{\HH}{\mathscr{H}}
\newcommand{\Om}{\Omega}
\newcommand{\Ho}{\in\HH(\Om)}
\newcommand{\bd}{\partial}
\newcommand{\del}{\partial}
\newcommand{\bardel}{\overline\partial}
\newcommand{\textdf}[1]{\textbf{\textsf{#1}}\index{#1}}
\newcommand{\img}{\mathrm{img}}
\newcommand{\ip}[2]{\left\langle{#1},{#2}\right\rangle}
\newcommand{\inter}[1]{\mathrm{int}{#1}}
\newcommand{\exter}[1]{\mathrm{ext}{#1}}
\newcommand{\cl}[1]{\mathrm{cl}{#1}}
\newcommand{\ds}{\displaystyle}
\newcommand{\vol}{\mathrm{vol}}
\newcommand{\cnt}{\mathrm{ct}}
\newcommand{\osc}{\mathrm{osc}}
\newcommand{\LL}{\mathbf{L}}
\newcommand{\UU}{\mathbf{U}}
\newcommand{\support}{\mathrm{support}}
\newcommand{\AND}{\;\wedge\;}
\newcommand{\OR}{\;\vee\;}
\newcommand{\Oset}{\varnothing}
\newcommand{\st}{\ni}
\newcommand{\wh}{\widehat}

%Pagination stuff.
\setlength{\topmargin}{-.3 in}
\setlength{\oddsidemargin}{0in}
\setlength{\evensidemargin}{0in}
\setlength{\textheight}{9.in}
\setlength{\textwidth}{6.5in}
\pagestyle{empty}


\begin{document}

\begin{center}
{\Large Data 227 - Autumn 2023 \hspace{0.5cm} HW 3}\\
\textbf{Data Visualization and Communication - Trimble}\\ %You should put your name here
Due: Friday October 27, 2023  11:59pm   
\end{center}

\vspace{0.2 cm}

\begin{enumerate}
\subsection*{Misleading visualization design pair}
\item\label{mislead}

For this assignment, you will create two static visualizations of a dataset of your choice:   a visualization engineered to obscure some truth of the data; and a visualization with an “honest” presentation of the same data.

All of the tools in our toolbox can be used well or poorly, and practicing using them to mislead will make us more savvy interpreters of data.
Some datasets that may lend themselves to deception:

\begin{enumerate}
\item Centers for Medicare and Medicaid Open Payments database (pharma compensation to prescribers, dubbed by Pro publica “Dollars for Docs”)
\item Guttmacher Institute Public-Use Datasets on Abortion and Fertility by Geography and Year
\item Indicators of Gender Equality from the World Bank 1960-2017
\item Tidy Tuesday Data Repository–multiple datasets for different weeks, going back to 2018 (can also be used with Python, all data in .csv format)
\item Washington Post’s DEA Pain Pills Database
\end{enumerate}

Your obfuscation should be free of catastrophic flaws (like axes that have no meaningful labels) and should be technically accurate (don't fake the data) but should lead readers to a conclusion about the data that is false.  (Make visualizations, find a fact in the data, make visualization to highlight the fact; negate the fact, make a new visualization, goose the visualization to make it misleading.)  

Your project submission should include:
\begin{enumerate}
\item A figure caption for each graphic, which needs to include provenance information sufficient to identify and locate the dataset.
\item A paragraph describing the techniques used to obfuscate the data.
\end{enumerate}

You do not have to use any specific tools to produce the visualization.  Submit the graphs and captions as a PDF, and the relevant code to produce the graph in a separate file.

%\begin{soln}
%	% Put your answers here.
%\end{soln}

\end{enumerate}

\end{document}


