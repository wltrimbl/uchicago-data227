\documentclass[]{book}

\usepackage[colorlinks=true, urlcolor=blue, linkcolor=red]{hyperref}

%These tell TeX which packages to use.
\usepackage{array,epsfig}
\usepackage{amsmath}
\usepackage{amsfonts}
\usepackage{amssymb}
\usepackage{amsxtra}
\usepackage{amsthm}
\usepackage{mathrsfs}
\usepackage{color}
\usepackage{enumitem}
%Here I define some theorem styles and shortcut commands for symbols I use often
\theoremstyle{definition}
\newtheorem{defn}{Definition}
\newtheorem{thm}{Theorem}
\newtheorem{cor}{Corollary}
\newtheorem*{rmk}{Remark}
\newtheorem{lem}{Lemma}
\newtheorem*{joke}{Joke}
\newtheorem{ex}{Example}
\newtheorem*{soln}{Solution}
\newtheorem{prop}{Proposition}

\newcommand{\lra}{\longrightarrow}
\newcommand{\ra}{\rightarrow}
\newcommand{\surj}{\twoheadrightarrow}
\newcommand{\graph}{\mathrm{graph}}
\newcommand{\bb}[1]{\mathbb{#1}}
\newcommand{\Z}{\bb{Z}}
\newcommand{\Q}{\bb{Q}}
\newcommand{\R}{\bb{R}}
\newcommand{\C}{\bb{C}}
\newcommand{\N}{\bb{N}}
\newcommand{\M}{\mathbf{M}}
\newcommand{\m}{\mathbf{m}}
\newcommand{\MM}{\mathscr{M}}
\newcommand{\HH}{\mathscr{H}}
\newcommand{\Om}{\Omega}
\newcommand{\Ho}{\in\HH(\Om)}
\newcommand{\bd}{\partial}
\newcommand{\del}{\partial}
\newcommand{\bardel}{\overline\partial}
\newcommand{\textdf}[1]{\textbf{\textsf{#1}}\index{#1}}
\newcommand{\img}{\mathrm{img}}
\newcommand{\ip}[2]{\left\langle{#1},{#2}\right\rangle}
\newcommand{\inter}[1]{\mathrm{int}{#1}}
\newcommand{\exter}[1]{\mathrm{ext}{#1}}
\newcommand{\cl}[1]{\mathrm{cl}{#1}}
\newcommand{\ds}{\displaystyle}
\newcommand{\vol}{\mathrm{vol}}
\newcommand{\cnt}{\mathrm{ct}}
\newcommand{\osc}{\mathrm{osc}}
\newcommand{\LL}{\mathbf{L}}
\newcommand{\UU}{\mathbf{U}}
\newcommand{\support}{\mathrm{support}}
\newcommand{\AND}{\;\wedge\;}
\newcommand{\OR}{\;\vee\;}
\newcommand{\Oset}{\varnothing}
\newcommand{\st}{\ni}
\newcommand{\wh}{\widehat}

%Pagination stuff.
\setlength{\topmargin}{-.3 in}
\setlength{\oddsidemargin}{0in}
\setlength{\evensidemargin}{0in}
\setlength{\textheight}{9.in}
\setlength{\textwidth}{6.5in}
\pagestyle{empty}


\begin{document}

\begin{center}
{\Large Data 227 - Autumn 2023 \hspace{0.5cm} HW 5}\\
\textbf{Data Visualization and Communication - Trimble}\\ %You should put your name here
Due: Monday November 13, 2023  11:59pm   
\end{center}

\vspace{0.2 cm}

\begin{enumerate}

\item
\subsection*{Stacked bar charts + color}

Freedom House is an organization that tracks changes in political rights and civil liberties in the world, publishing per-country and per-region ratings on various dimensions in the "Freedom in the World" report.   Obtain the FIW data 2013-2023.  
(Gorokhovskaia, Shahbaz, and Slipowitz.  Freedom in the World 2023. Freedom in the World (2023).

\begin {enumerate} 
\item Suppose we are interested in seeing how civil liberties have changed in the pandemic era (for our purposes, defined as 2020-2022). First, read in the data and calculate the differences in total civil liberties scores from 2019 to 2022 (1 pt).
\item Using the calculated differences, create a chloropleth map showing the change for every country/territory. Make sure to use a meaningful color scheme using guidelines from the last few weeks–give a brief description of the scheme you used and a justification for using it. According to your chloropleth map, what are the global trends in changes in civil liberties in the pandemic era? (3 pts)
\item  Zoom into South America; change the Coordinate Reference System (CRS) of your graph to a projection suitable for the continent. (1 pt)
\end{enumerate}

\item
\subsection*{Choropleths Chicago}

The City of Chicago Data Portal publishes coded information about the calls to the city's information  and service request line.

Some of the complaints have geographical information, some do not; you should ignore requests that have catch-all addresses associated with them.

You may use any recent year or ranges of years as long as you indicate the scope of your reporting.  Your choropleths can use community areas or zip codes, whichever is more convenient for you to plot.  Try not to plot data artifacts.

Make a choropleth maps to answer the following questions about 311 calls in Chicago.  
\begin{enumerate}
\item  Which neighborhoods have the most graffiti removal requests?  
\item  Can you find and graph a complaint category with the most nearly equal per-capita rates?  (What issues seem to affect people in the city at the same rates?)
\item  Can you find and graph a complaint category with the most unequal per-capita rates? 
\item  Can you find an interesting (geographical) fact in the 311 calls dataset?

\end{enumerate}

%\begin{soln}
%	% Put your answers here.
%\end{soln}

\end{enumerate}

\end{document}


