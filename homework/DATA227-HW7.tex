\documentclass[]{book}

\usepackage[colorlinks=true, urlcolor=blue, linkcolor=red]{hyperref}

%These tell TeX which packages to use.
\usepackage{array,epsfig}
\usepackage{amsmath}
\usepackage{amsfonts}
\usepackage{amssymb}
\usepackage{amsxtra}
\usepackage{amsthm}
\usepackage{mathrsfs}
\usepackage{color}
\usepackage{enumitem}
%Here I define some theorem styles and shortcut commands for symbols I use often
\theoremstyle{definition}
\newtheorem{defn}{Definition}
\newtheorem{thm}{Theorem}
\newtheorem{cor}{Corollary}
\newtheorem*{rmk}{Remark}
\newtheorem{lem}{Lemma}
\newtheorem*{joke}{Joke}
\newtheorem{ex}{Example}
\newtheorem*{soln}{Solution}
\newtheorem{prop}{Proposition}

\newcommand{\lra}{\longrightarrow}
\newcommand{\ra}{\rightarrow}
\newcommand{\surj}{\twoheadrightarrow}
\newcommand{\graph}{\mathrm{graph}}
\newcommand{\bb}[1]{\mathbb{#1}}
\newcommand{\Z}{\bb{Z}}
\newcommand{\Q}{\bb{Q}}
\newcommand{\R}{\bb{R}}
\newcommand{\C}{\bb{C}}
\newcommand{\N}{\bb{N}}
\newcommand{\M}{\mathbf{M}}
\newcommand{\m}{\mathbf{m}}
\newcommand{\MM}{\mathscr{M}}
\newcommand{\HH}{\mathscr{H}}
\newcommand{\Om}{\Omega}
\newcommand{\Ho}{\in\HH(\Om)}
\newcommand{\bd}{\partial}
\newcommand{\del}{\partial}
\newcommand{\bardel}{\overline\partial}
\newcommand{\textdf}[1]{\textbf{\textsf{#1}}\index{#1}}
\newcommand{\img}{\mathrm{img}}
\newcommand{\ip}[2]{\left\langle{#1},{#2}\right\rangle}
\newcommand{\inter}[1]{\mathrm{int}{#1}}
\newcommand{\exter}[1]{\mathrm{ext}{#1}}
\newcommand{\cl}[1]{\mathrm{cl}{#1}}
\newcommand{\ds}{\displaystyle}
\newcommand{\vol}{\mathrm{vol}}
\newcommand{\cnt}{\mathrm{ct}}
\newcommand{\osc}{\mathrm{osc}}
\newcommand{\LL}{\mathbf{L}}
\newcommand{\UU}{\mathbf{U}}
\newcommand{\support}{\mathrm{support}}
\newcommand{\AND}{\;\wedge\;}
\newcommand{\OR}{\;\vee\;}
\newcommand{\Oset}{\varnothing}
\newcommand{\st}{\ni}
\newcommand{\wh}{\widehat}

%Pagination stuff.
\setlength{\topmargin}{-.3 in}
\setlength{\oddsidemargin}{0in}
\setlength{\evensidemargin}{0in}
\setlength{\textheight}{9.in}
\setlength{\textwidth}{6.5in}
\pagestyle{empty}


\begin{document}

\begin{center}
{\Large Data 227 - Autumn 2023 \hspace{0.5cm} HW 7}\\
\textbf{Data Visualization and Communication - Trimble}\\ %You should put your name here
Due: Monday Dec 4, 2023  11:59pm    (after final project!)
\end{center}

\vspace{0.2 cm}

\begin{enumerate}

\item
\subsection*{Draw some error bars }
The American Community Survey is a major sampling campaign by the U. S. Census Bureau which collects a wide variety of statistics about people who live in the United States.   For some statistics, the Census publishes uncertainty estimates (conventionally 90\% error bands) for their estimates.

You can get data from the firehose at:
\texttt{https://data.census.gov/table}

Make a scatterplot of two statistics (mean income vs. mean age, or mean years education vs. mean ages) from the ACS where each point is a county.  Indicate the error bands in both dimensions on the graph.  
Use color to encode county population. 


\item
\subsection*{Network diagram}
Make a visualization (static or interactive) that commuincates something about a network dataset of your choice.  Include a full caption that explains the origin of the dataset.

\item
\subsection*{Interaction}
Make a graph using a dataset of your choice that includes at least 50 data points and is interactive.  Use a mode of interaction other than hover-over text (you are welcome and encouraged to use hover-over, but make sure you use something else too), such as brushing, subset-selection, or animation of some sort.  If appropriate, you can use hyperlinks in your graph.

To share an interactive graph, you will need to export your graph as html, and should use a software package that embeds javascript containing the data into html.   Altair has a limit of 5000 rows per datastructure; you probably shouldn't use more than 5000 rows for plotly either; it is best to choose a dataset (or subset a dataset) until you have fewer than 5000 things to display.  

Since this is a dataset of your choice, please write a full caption that indicates where the dataset is from and what the dataset shows.

%\begin{soln}
%	% Put your answers here.
%\end{soln}

\end{enumerate}

\end{document}


